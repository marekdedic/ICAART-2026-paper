\section{\uppercase{Problem Statement}}

As noted in Section~\ref{sec:related-work}, the existing works evaluating hyper-parameter optimization (HPO) methods are limited in scope. They typically focus on a small number of graph datasets, usually in a single domain, a limited set of hyper-parameters, or a limited set of HPO methods. This makes it difficult to draw general conclusions about the performance of different HPO methods across a wide range of scenarios. One of the primary goals of this work is to address these limitations by conducting a comprehensive evaluation of various HPO methods across multiple graph datasets from different domains, considering a broader set of hyper-parameters and methods.

Additionally, we propose a novel approach to HPO on graph datasets specifically. This approach leverages the fact that graph datasets have an extensive internal structure that can be characterized by various statistical properties (e.g., number of nodes, edges, degree distribution, homophily, etc.). Unlike traditional tabular, image or text datasets, these properties can provide valuable insights into the nature of the data and its underlying relationships. Our work aims to exploit these unique characteristics of graph datasets to enhance the HPO process. Specifically, we argue that these graph-specific properties can be used to inform and guide the HPO process more effectively than generic methods that do not consider the structure of the data.
